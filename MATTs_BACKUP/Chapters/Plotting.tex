	\subsection{Plotting} \label{sec:Plotting} \index{Visualization|see{Plotting}}
			As explained in the previous section, simulation data can be visualized \index{Visualization} through the {\it gui\_database} window, but also manually.
			In this section the plotting \index{Plotting} functions are explained.
			
			\lstinline{function plotBandGap(QDOA)} \\ \\
				
			
			\lstinline{function plotEnergyLevels(QDOA)} \\ \\
			\lstinline{function plotPL(EDOA)} \\ \\
				 Plots the photoluminescence of an array of experimental data. \\ \\
			\lstinline{function plotAbs(EDOA)} \\ \\
				 Plots the absorption spectrum of an array of experimental data. \\ \\
			\lstinline{function plotQDStructure(QDOA)} \\ \\
			\lstinline{function plotVoltBandGap(QDOA)} \\ \\
			
	\subsection{Experimental data}
		Working with experimental data goes along similarly to the handling of simulation data. The class {\it ExpData} defines properties, such as material name, quantum
		dot sizes measured from \glspl{TEM}, \glspl{PL}- or absorption spectrums. The so called \glspl{EDO} contain data of one single experiment. With these \glspl{EDO}
		the \gls{PL} or absorption spectrum can be plotted. Please see \lstinline{function plotPL(EDOA)} and \lstinline{function plotAbs(EDOA)} above.		
			
	\subsection{Additional tools} \label{sec:addTools}
			\lstinline{function [EDOA, ExpDataDirs] = getEDOA()} \\ \\
				This will return all experimental data sets in form of a EDOA and additionally the files, where the data are stored.
				
			\lstinline{function [BGap, Radius, Volt, Mat] = getBandGap(QDO)} \\ \\
				This function returns the bandgap in $eV$, radius in $nm$, applied voltage in $V$ and the material as a number specified in {\it gui\_simulate}
				(i.e. 1 for PbSe\_allan) of a single QDO. \\ \\
			
			\lstinline{function updateProperty(DIR, propertyName, value)} \\ \\
				Changes the property with {\it propertyName} with the {\it value} of every QDO stored in the directory {\it DIR}. The directory path is
				relative to \lstinline{'root/Simulations'}. The value has to have the same type as the property, that will be changed. \\ \\